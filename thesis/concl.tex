%! TEX root = thesis.tex
\chapter{Conclusion}\label{sec:concl}

We present \algo, a collection of algorithms for sampling unlabeled
transactional datasets from a null model, which is a key step in evaluating the
statistical validity of the results of knowledge discovery tasks. \algo\ uses
swap-randomization, a Markov-Chain-Monte-Carlo approach. In contrast with
previous work, \algo\ does not suffer from a distortion of the space of
datasets, which was due to an assumed 1-to-1 mapping of the datasets to binary
matrices. This distortion may (and does) affect the outcome of the validation,
i.e., results in false discoveries. We avoid this assumption and show two
methods (\naivealgo\ and \refalgo) for drawing datasets from the non-distorted
null space. Our experimental evaluation shows that the distortion is relevant,
but \algo\ does not suffer from it, is fast, and scales well as the dataset
grows.

Interesting directions for future work include the definition of richer null
models that capture more important properties of the data, and fast algorithms
to sample from these null models.

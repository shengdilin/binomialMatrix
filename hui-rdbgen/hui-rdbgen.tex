\documentclass{article}

\usepackage[utf8]{inputenc}
\usepackage{amsmath}
\usepackage{amsthm}

\newtheorem{theorem}{Theorem}[section]
\newtheorem{corollary}{Corollary}[theorem]
\newtheorem{lemma}[theorem]{Lemma}

\title{Statistically Significant High Utility Itemset Mining --- Random
Database Generation}
\date{}
\author{}

\begin{document}
\maketitle

\section{Notation}

\begin{itemize}
    \item $I \doteq \{i_1, \dots, i_{|I|}\}$: set of all items $I$.
    \item $D \doteq \{T_1, \dots, T_{|D|}\}$: quantitative transaction database
        $D$.
    \item $T_c \subseteq I$: transaction $T_c$ that has a unique identifier
        $c$.
    \item $X \subseteq I$: itemset $X$.
    \item $p_D(i)$: external utility (i.e., price) of item $i$.
    \item $q(i, T_c)$: internal utility (i.e., quantity) of item $i$ in
        transaction $T_c$.
    \item $q(T_c) \doteq \sum_{i \in T_c} q(i, T_c)$: total internal utility of
        a transaction $T_c$ in database $D$.
    \item $q_D(i) \doteq \sum_{T_c \in D} q(i, T_c)$: total internal utility of
        item $i$ in database $D$.
    \item $u(i, T_c) \doteq p_D(i) \cdot q(i, T_c) $: utility of item $i$ in
        transaction $T_c$.
    \item $u(X, T_c) \doteq \sum_{i \in X} u(i, T_c)$: utility of an itemset
        $X$ in transaction $T_c$.
    \item $u_D(X) \doteq \sum_{T_c \in D} u(X, T_c)$: utility of itemset $X$ in
        database $D$.
    \item $TU(T_c) \doteq \sum_{i \in T_c} u(i, T_c) = u(T_c, T_c)$:
        transaction utility $TU$ of transaction $T_c$.
    \item $s_D(X) \doteq |\{T_c \in D : X \subseteq T_c\}|$: the support of
        itemset $X$ in database $D$.
\end{itemize}

\section{Database Properties to Preserve}

\begin{enumerate}
    \item\label{prop:itemextutil} The external utility of each item $i$
        in database $D$, $p_D(i)$. This can be thought of as preserving the
        price of each item in the e-commerce store.
    \item\label{prop:itemtotintutil} The total internal utility of each item
        $i$ in database $D$, $q_D(i)$. This can be thought of as preserving the
        total purchased quantity of an item in the e-commerce store.
    \item\label{prop:disttranslen} The distribution of transaction lengths in
        database $D$, $\{|T_c| : T_c \in D\}$. This can be thought of as
        preserving the number of unique items bought for each transaction in
        the e-commerce store.
    \item\label{prop:disttransutil} The distribution of the transaction
        utilities in database $D$, $\{TU(T_c) : T_c \in D\}$. This can be
        thought of as preserving the amount of profit generated from each
        transaction in the e-commerce store.
    \item\label{prop:disttransintutil} The distribution of the transaction
        internal utilities in database $D$, $\{q(T_c) : T_c \in D\}$. This can
        be thought of as preserving the quantity of items bought for each
        transaction in the e-commerce store.
    \item\label{prop:distitemintutil} The distribution of internal utilities
        for each item across the transactions in database $D$, $\{q(i, T_c) : i
        \in T_c, \ T_c \in D\} \ \forall i \in I$. This can be thought of
        as preserving the purchased quantities of an item for each transaction
        in the e-commerce store.
\end{enumerate}

\section{Random Database Generation Methods}

For the following methods, we will always preserve Prop.~\ref{prop:itemextutil}
since doing so is trivial.

\begin{enumerate}
    \item For each item $i \in I$, randomly repartition $q_D(i)$ into $\leq
        |D|$ utilities. Randomly add $i$ and each utility into a unique
        transaction in the new database $D^\prime$. This method preserves
        Prop.~\ref{prop:itemextutil} and Prop.~\ref{prop:itemtotintutil}.
    \item\label{method:disttranslenswaps} Randomly select two transactions
        $T_c$ and $T_d$ with probability $\Pr(c) = \frac{|T_c|}{\sum_{j =
        1}^{|D|} |T_j|}$ and $\Pr(d) = \frac{|T_d|}{\sum_{j = 1}^{|D|} |T_j|}$,
        respectively. Then, select items $i_k \in T_c$ and $i_l \in T_d$
        uniformly at random within each transaction. If $i_k \notin T_d$ and
        $i_l \notin T_c$, then swap $i_k$ and $q_D(i_k, T_c)$ with $i_l$ and
        $q_D(i_l, T_d)$. Continue such execution for $n$ steps. This method
        preserves Prop.~\ref{prop:itemextutil} and
        Prop.~\ref{prop:disttranslen}.
    \item Execute Method~\ref{method:disttranslenswaps}. Then, for each item $i
        \in I$, randomly repartition $q_D(i)$ into $s_D(\{i\})$ utilities.
        Place the repartitioned internal utilities back into the transactions
        that contain $i$. This method preserves Prop.~\ref{prop:itemextutil},
        Prop.~\ref{prop:itemtotintutil}, and Prop.~\ref{prop:disttranslen}.
    \item Execute Method~\ref{method:disttranslenswaps}.  Then, for each item
        $i \in I$, construct a list $[q(i, T_c) : i \in T_c, \ T_c \in D]$.
        Permute the list and place the new ordering of internal utilities back
        into the transactions that contain $i$. This method preserves
        Prop.~\ref{prop:itemextutil}, Prop.~\ref{prop:disttranslen}, and
        Prop.~\ref{prop:distitemintutil}.
    \item\label{method:disttransutilswaps} Create a set of lists $U = \{L_j :
        L_j = [(i, T_c) : u(i, T_c) = j, \ i \in I, \ T_c \in D]\}$, where the
        lists in $U$ contain item-transaction tuples of equal utility.  Select
        a list $L_j \in U$ with probability $\Pr(j) = \frac{{|L_j|}^2}{\sum_{k
        = 1}^m {|L_k|}^2}$, where $m = \max\{u(i, T_c) : i \in I,\  T_c \in
        D\}$, to ensure that item pairs are selected uniformly at random. Then,
        randomly select $(i_k, T_c), (i_l, T_d) \in L_j$ such that $i_k \notin
        T_d$ and $i_l \notin T_c$. Next, swap $(i_k, T_c)$ and $(i_l, T_d)$.
        Continue such execution for $n$ steps. This method preserves
        Prop.~\ref{prop:itemextutil}, Prop.~\ref{prop:itemtotintutil},
        Prop.~\ref{prop:disttranslen}, Prop.~\ref{prop:disttransutil}, and
        Prop.~\ref{prop:distitemintutil}.
    \item Execute Method~\ref{method:disttransutilswaps} but with internal
        utilities instead of utilities. This method preserves
        Prop.~\ref{prop:itemextutil}, Prop.~\ref{prop:itemtotintutil},
        Prop.~\ref{prop:disttranslen}, Prop.~\ref{prop:disttransintutil},
        and Prop.~\ref{prop:distitemintutil}.
\end{enumerate}

\section{Theorems}

\begin{theorem}
    A random walk on a directed graph $G$ converges to a stationary
    distribution $\bar{\pi}$, where \[\pi_v = \frac{d(v)}{|E|}.\]
\end{theorem}

\begin{proof}
    Since $\sum_{v \in V} d(v) = |E|$, it follows that \[\sum_{v \in V} \pi_v
    = \sum_{v \in V} \frac{d(v)}{|E|} = 1,\] and $\bar{\pi}$ is a proper
    distribution over $v \in V$.

    Let $\mathbf{P}$ be the transition probability matrix of the Markov chain.
    Let $N(v)$ represent the neighbors of $v$. The relation $\bar{\pi} =
    \bar{\pi} \mathbf{P}$ is equivalent to \[\pi_v = \sum_{u \in N(v)}
    \frac{d(u)}{|E|} \frac{1}{d(u)} = \frac{d(v)}{|E|},\] and the theorem
    follows.
\end{proof}

\begin{theorem}
    The stationary distribution of an irreducible aperiodic finite Markov chain
    is uniform if and only if its transition matrix is doubly stochastic.
\end{theorem}

\begin{corollary}
    If the transition matrix of an irreducible aperiodic finite Markov chain is
    symmetric, then the stationary distribution of the Markov chain is uniform.
\end{corollary}

\begin{proof}
    Since the transition matrix is symmetric and its rows sum to 1, its columns
    also sum to 1. Thus, the transition matrix is doubly stochastic and so the
    stationary distribution is uniform.
\end{proof}

\begin{lemma}[Metropolis-Hastings Algorithm]
    Suppose a Markov chain on a finite state space $\Omega$ is given by the
    transition matrix $\mathbf{Q}$ and a neighborhood structure $\{N(x) : x \in
    \Omega\}$.  For all $x \in \Omega$, let $\pi_x$ be the probability of state
    $x$ in the stationary distribution of the desired Markov chain. Consider
    the Markov chain where
    \[
    P_{x, y} = \begin{cases}
        Q_{x, y} \min\left(\frac{\pi_y Q_{y, x}}{\pi_x Q_{x, y}}, 1\right) &
        \text{ if } x \neq y \text{ and } y \in N(x). \\
        0 & \text{ if } x \neq y \text{ and } y \notin N(x). \\
        1 - \sum_{y \neq x} P_{x, y} & \text{ if } x = y.
    \end{cases}
    \]
    Then, if this chain is irreducible and aperiodic, the stationary
    distribution is given by the probabilities $\pi_x$.
\end{lemma}

\begin{proof}
    We show that the chain is time reversible and thus has a stationary
    distribution given by $\pi_x$. For any $x \neq y$, if $\pi_x Q_{x, y} \leq
    \pi_y Q_{y, x}$, then $P_{x, y} = Q_{x, y}$ and $P_{y, x} = \frac{\pi_x
    Q_{x, y}}{\pi_y}$.  It follows that $\pi_x P_{x, y} = \pi_y P_{y, x}$.
    Similarly, if $\pi_x Q_{x, y} > \pi_y Q_{y, x}$, then $P_{x, y} =
    \frac{\pi_y Q_{y, x}}{\pi_x}$ and $P_{y, x} = Q_{y, x}$, and it follows
    that $\pi_x P_{x, y} = \pi_y P_{y, x}$. By Theorem 7.10 in the textbook,
    the stationary distribution is given by the values $\pi_x$.
\end{proof}

\end{document}

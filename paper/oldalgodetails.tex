\newcommand{\bijfunc}{f} % bijective function
\newcommand{\coledges}[2]{C_{#1 #2}} % {j : (#1, j), (#2, j) in edges}
\newcommand{\colsneqnum}[2]{\eta_{#1, #2}} % number of column values that are not equal across rows #1 and #2
\newcommand{\ddt}{\ddtsym} % product of dataset matrix and its transposition
\newcommand{\ddtadj}{\ddtsym^\prime} % product of adjacent dataset matrix and its transposition
\newcommand{\ddtdel}[2]{\mu(#1, #2)} % [M'(#1, #2)^2 - M(#1, #2)^2] - [M'(#1, #2) - M(#1, #2)]
\newcommand{\ddtsym}{M} % symbol for the product of dataset matrix and its transposition
\newcommand{\degreedel}{\Delta\degreesym} % change in the degree
\newcommand{\degree}[1]{\degreesym(#1)} % degree of a state in a Markov chain (number of out-neighbors)
\newcommand{\degreesym}{\mathsf{deg}} % symbol for degree of graph
\newcommand{\edges}{\edgessym} % set of edges in bipartite graph
\newcommand{\edgesnew}{\edgessym^\newsym} % new set of edges in bipartite graph
\newcommand{\edgessym}{E} % symbol for set of edges
\newcommand{\eqswaprowselem}[1]{\ell_{#1}} % an element in the set of row tuples (u, w) such that rows u, w equal the rows involved in the swap
\newcommand{\eqgraphs}{\Psi} % set of equivalent graphs yielded from a graph with a local swap defined by Gionis et al.
\newcommand{\eqgraphselem}[1]{\psi_{#1}} % an element in the set of equivalent graphs yielded from a graph with a local swap defined by Gionis et al.
\newcommand{\eqgraphsnum}[1]{\eqgraphsnumsym(#1)} % number of equivalent graphs yielded from #1 with a local swap defined by Gionis et al.
\newcommand{\eqgraphsnumdel}{\Delta\eqgraphsnumsym} % change in the number of equivalent graphs yielded from a graph with a local swap defined by Gionis et al.
\newcommand{\eqgraphsnumsym}{H} % symbol for the number of equivalent graphs yielded from a graph with a local swap defined by Gionis et al.
\newcommand{\eqswaprows}{\mathcal{L}} % set of transactions pairs (u, w) equal to the transactions involved in the swaps.
\newcommand{\esupp}[1]{\mu\lparen#1\rparen} % expected support of itemset #1 under the null model (implicit)
\newcommand{\findadj}{\texttt{findAdjacent}} % findAdjacent method
\newcommand{\findadjrev}{\texttt{findAdjacentRevised}} % findAdjacentRevised method
\newcommand{\geteqgraphsnumdel}{\texttt{get$\eqgraphsnumdel$}} % method to get change in the number of equivalent graphs yielded from a graph with a local swap defined by Gionis et al.
\newcommand{\graph}{\graphsym} % bipartite graph representation for dataset
\newcommand{\graphadj}{\graphsym^\newsym} % adjacent bipartite graph
\newcommand{\graphadjnum}[1]{\tau(#1)} % total number of not equivalent adjacent graphs from graph #1
\newcommand{\graphsym}{G} % symbol for bipartite graph
\newcommand{\kttsnumdel}{\Delta\kttnumsym} % change in number of K_{2, 2} cliques
\newcommand{\markchain}{\mathcal{Q}} % Markov chain
\newcommand{\methas}{\texttt{metropolisHastings}} % metropolisHastings method
\newcommand{\methasrev}{\texttt{metropolisHastingsRevised}} % metropolisHastingsRevised method
\newcommand{\miningalgo}{\mathsf{A}} % frequent itemset mining algorithm
\newcommand{\naive}{\texttt{na\"{\i}ve}} % naive method
\newcommand{\result}{\resultsym} % result of frequent itemset mining algorithm
\newcommand{\results}{\mathbf{\resultsym}} % set of results
\newcommand{\resultsym}{X} % symbol for result
\newcommand{\rnfi}[1]{RNFI(#1)} % relative number of frequent itemsets for the sampled dataset #1 % chktex 36
\newcommand{\selfloop}{\texttt{selfLoop}} % selfLoop method
\newcommand{\spairsgi}{Y} % set of swappable pairs of edges defined by Gionis et al.
\newcommand{\spairsgieq}{\Phi} % set of swappable pairs of edges defined by Gionis et al. that lead to an equivalent graph
\newcommand{\spairsgieqelem}[1]{\phi_{#1}} % an element in the set of swappable pairs of edges defined by Gionis et al. that lead to an equivalent graph
\newcommand{\spairselem}[1]{g_{#1}} % an element in the set of swaps from a dataset to another
\newcommand{\statprobnew}{\statprob'} % new stationary distribution
\newcommand{\tprobmatn}{\mathbf{\tprobmatnsym}} % new transition probability matrix
\newcommand{\tprobmatnsym}{P} % symbol for new transition probability matrix
\newcommand{\tprobmatnv}[2]{\tprobmatnsym_{#1, #2}} % value of new transition probability matrix at #1, #2
\newcommand{\tprobmato}{\mathbf{\tprobmatosym}} % original transition probability matrix
\newcommand{\tprobmatosym}{Q} % symbol for original transition probability matrix
\newcommand{\tprobmatov}[2]{\tprobmatosym_{#1, #2}} % value of original transition probability matrix at #1, #2
\newcommand{\transitions}{\mathcal{T}} % set of transitions (pairs of graphs)
\newcommand{\zstructsnumdel}{\Delta\zstructsnumsym} % change in number of Z structures
\section{Algorithms}

In order to develop methods to sample datasets uniformly at random from the
distribution of transactional datasets that have the same row and column
margins, we begin with formalizing the notation for equivalent transactional
datasets.

Let $\dataset$ and $\datasetadj$ be transactional (0--1) datasets. We denote
the relation $\equiv$ by $\dataset \equiv \datasetadj$ if and only if
$\dataset$ and $\datasetadj$ are equivalent up to the order of transactions.
Similarly, we write $\graph \equiv \graphadj$ if and only if $\dataset \equiv
\datasetadj$, where $\graph$ and $\graphadj$ are the graphs associated to
$\dataset$ and $\datasetadj$, respectively.

Now, we provide a new condition for the local swap so that $\dataset \not\equiv
\datasetadj$. Given $(i, j), (k, l) \in \edges(\graph)$, not only must (1) $(i,
l), (k, j) \notin \edges(\graph)$, but it also must be that (2) there exists $c
\in \cols$ such that $\dataset(i, c) \neq \dataset(k, c)$.
Lemma~\ref{lem:notequiv} shows why these conditions must be true, and
Algorithm~\ref{algo:findadjrev} details a revised \findadj{} method that
incorporates condition (2). From now on, we will refer to pairs of edges that
satisfy the two conditions above as \textit{swappable}.

\begin{lemma}\label{lem:notequiv}
  Suppose we obtained $\graphadj$ from $\graph$ via the following swap:
  \[
    \edges(\graphadj) \leftarrow \edges(\graph) \setminus \{(i, j), (k, l)\}
    \cup \{(i, l), (k, j)\},
  \]
  where $(i, j), (k, l) \in \edges(\graph)$ and $(i, l), (k, j) \notin
  \edges(\graph)$. Let $\dataset$ and $\datasetadj$ be the datasets associated
  to $\graph$ and $\graphadj$, respectively. Then, $\dataset \not\equiv
  \datasetadj$ if and only if there exists $c \in \cols \setminus \{j, l \}$
  such that $\dataset(i, c) \neq \dataset(k, c)$.
\end{lemma}

\begin{proof}
  We begin with proving the forward direction by proving its contrapositive. If
  for all $c \in \cols \setminus \{j, l\}, \dataset(i, c) = \dataset(k, c)$,
  then rows $\dataset(i) = \datasetadj(k)$ and $\dataset(k) = \datasetadj(i)$.
  Because only rows $i, k$ are different and all other rows stayed the same
  after the swap, we have that $\dataset \equiv \datasetadj$.

  Next, we prove the backward direction directly. If there exists $c \in \cols
  \setminus \{j, l\}$ such that $\dataset(i, c) \neq \dataset(k, c)$, then rows
  $\dataset(i) \neq \datasetadj(k)$ and $\dataset(k) \neq \datasetadj(i)$.
  It follows that $\dataset \not\equiv \datasetadj$.
\end{proof}

\begin{algorithm}
	\caption{\findadjrev}\label{algo:findadjrev}
	\DontPrintSemicolon{}
	\KwIn{Graph $\graph$}
	\KwOut{Graph $\graphadj$ such that $(\graph, \graphadj) \in \transitions$}

  \Repeat{$(i, l), (k, j) \notin \edges(\graph)$ and $\exists c \in \cols
    \setminus \{j, l\}$ s.t. $\dataset(i, c) \neq \dataset(k, c)$}{
		Select edges $(i, j), (k, l) \in \edges(\graph)$ uniformly at random\;
	}
	$\edges(\graphadj) \leftarrow \edges(\graph) \setminus \{(i, j), (k, l)\}
	\cup \{(i, l), (k, j)\}$\;
	\Return{$\graphadj$}
\end{algorithm}

If we were to run the \naive{} method with \findadjrev{} instead of \findadj{},
then our transition probability matrix $\tprobmato$ would have transition
probabilities $\tprobmatov{\graph}{\graphadj} =
\frac{\spairsnum{\graph}{\graphadj}}{\graphadjnum{\graph}}$, where $\graph
\not\equiv \graphadj$, $\spairsnum{\graph}{\graphadj}$ is the number of
swappable pairs of edges that transition to state $\graphadj$ given that we are
at $\graph$, and $\graphadjnum{\graph}$ is the total number of swappable pairs
of edges for state $\graph$ (i.e., $\graphadjnum{\graph} = \sum_{\graphadj}
\spairsnum{\graph}{\graphadj}$). Like before, it is not difficult to show that
this revised version of the \naive{} method does not always converge to a
uniform stationary distribution.

\subsection{The Metropolis-Hastings Method Revised}

To construct a Markov chain where each graph (state) is different up to the
order of transactions and where the chain converges a uniform stationary
distribution, we leverage the \methas{} method by \citet{GionisMMT07}.
Here, our new transition probability matrix $\tprobmatn$ has transition
probabilities
\[
  \tprobmatnv{\graph}{\graphadj} =
  \frac{\spairsnum{\graph}{\graphadj}}{\graphadjnum{\graph}} \min\{1,
  \frac{\spairsnum{\graphadj}{\graph}
  \graphadjnum{\graph}}{\spairsnum{\graph}{\graphadj}
  \graphadjnum{\graphadj}}\}.
\]
That is, \findadjrev{} first selects an adjacent graph $\graphadj$ with
probability $\frac{\spairsnum{\graph}{\graphadj}}{\graphadjnum{\graph}}$. Then,
we transition to $\graphadj$ with an acceptance probability of $\min\{1,
\frac{\spairsnum{\graphadj}{\graph}
\graphadjnum{\graph}}{\spairsnum{\graph}{\graphadj} \graphadjnum{\graphadj}}\}$.
The revised \methas{} method is detailed in Algorithm~\ref{algo:methasrev}.

\begin{algorithm}
	\caption{\methasrev}\label{algo:methasrev}
	\DontPrintSemicolon{}
	\KwIn{Graph $\graph_\dataset$, number of random walk steps $\steps$}
	\KwOut{Graph $\graph$ with the same degree sequences as $\graph_\dataset$}

	$\graph \leftarrow \graph_\dataset$\;
	\While{$\steps > 0$}{
		$\graphadj \leftarrow \findadjrev(\graph)$\;
    $\graph \leftarrow \graphadj$, with probability $\min\{1,
    \frac{\spairsnum{\graphadj}{\graph}
    \graphadjnum{\graph}}{\spairsnum{\graph}{\graphadj}
    \graphadjnum{\graphadj}}\}$\;
    $\steps \leftarrow \steps - 1$\;
	}
	\Return{$\graph$}
\end{algorithm}

The main task in the \methasrev{} method is computing
$\spairsnum{\graph}{\graphadj}$, $\graphadjnum{\graph}$,
$\spairsnum{\graphadj}{\graph}$, and $\graphadjnum{\graphadj}$. We begin with
detailing how to compute $\spairsnum{\graph}{\graphadj}$ and
$\graphadjnum{\graph}$ in the following theorems.

\begin{theorem}
  Suppose we obtained $\graphadj$ from $\graph$ via the following swap:
  \[
    \edges(\graphadj) \leftarrow \edges(\graph) \setminus \{(i, j), (k, l)\}
    \cup \{(i, l), (k, j)\},
  \]
  where $(i, j), (k, l) \in \edges(\graph)$, $(i, l), (k, j) \notin
  \edges(\graph)$, and there exists $c \in \cols \setminus \{i, j\}$ such that
  $\dataset(i, c) \neq \dataset(k, c)$. Let $\dataset$ and $\datasetadj$ be the
  datasets associated to $\graph$ and $\graphadj$, respectively. Then, the
  number of swappable pairs of edges that transition to state $\graphadj$ given
  that we are at $\graph$ is
  \[
    \spairsnum{\graph}{\graphadj} = (\text{\# of rows equal to $\dataset(i)$})
    \cdot (\text{\# of rows equal to $\dataset(j)$}).
  \]
\end{theorem}

\begin{proof}
  We begin by defining the following sets.
  \begin{align*}
    \spairs &= \{((u, v), (w, x)) : (u, v), (w, x) \in \edges(\graph), u < w
    \wedge \text{swapping $(u, v), (w, x)$ transitions to $\graphadj$}\}, \\
    \eqswaprows &= \{(u, w) : u, w \in \rows, u < w \wedge \dataset(u) =
    \dataset(i) \wedge \dataset(w) = \dataset(k)\}.
  \end{align*}
  It suffices to show that $|\spairs| = |\eqswaprows|$. To do so, we find a
  bijective function mapping from $\spairs$ to $\eqswaprows$.  Let $\bijfunc :
  \spairs \rightarrow \eqswaprows$ be a function defined by
  \[
    \bijfunc((u, v), (w, x)) = (u, w),
  \]
  where $((u, v), (w, x)) \in \spairs$ and $(u, w) \in \eqswaprows$.

  To prove that $\bijfunc$ is bijective, we first show that $\bijfunc$ is
  injective. Let $\spairselem{1}, \spairselem{2} \in \spairs$, where
  $\spairselem{1} = ((u_1, v_1), (w_1, x_1))$ and $\spairselem{2} = ((u_2,
  v_2), (w_2, x_2))$. If $\bijfunc(\spairselem{1}) = \bijfunc(\spairselem{2})$,
  then $(u_1, w_1) = (u_2, w_2)$, and thus $u_1 = u_2$ and $w_1 = w_2$. If we
  swap a pair of edges across rows $u_1 = u_2$ and $w_1 = w_2$ in $\dataset$
  such that the swap generates $\datasetadj$, there can only be two columns
  involved in the swap. As such, we have that $v_1 = v_2$ and $w_1 = w_2$. It
  follows that $\spairselem{1} = \spairselem{2}$, and so $\bijfunc$ is
  injective.

  Next, we show that $\bijfunc$ is surjective. Let $\eqswaprowselem{} \in
  \eqswaprows$, where $\eqswaprowselem{} = (u, w)$. Hence, $\dataset(u) =
  \dataset(i)$ and $\dataset(w) = \dataset(k)$, implying that swapping $(u, j)$
  and $(w, l)$ also transitions to $\graphadj$. If $\spairselem{} \in \spairs$
  such that $\spairselem{} = ((u, j), (w, l))$, it follows that $\bijfunc$ is
  surjective because $\bijfunc(\spairselem{}) = \eqswaprowselem{}$.

  Having shown that $\bijfunc$ is both injective and surjective, $\bijfunc$ is
  therefore bijective and $|\spairs| = |\eqswaprows|$.
\end{proof}

\begin{theorem}\label{thm:graphadjnum}
  The total number of swappable pairs of edges for a state $\graph$ is equal to
  \begin{equation}\label{eq:graphadjnum}
    \graphadjnum{\graph} = \degree{\graph} - \eqgraphsnum{\graph},
  \end{equation}
  where, as in Theorem~\ref{thm:degree}, $\degree{\graph}$ is the number of
  graphs $\graphadj$ that are yielded from $\graph$ with a local swap defined
  by Algorithm~\ref{algo:findadj} \textnormal{\findadj{}}, and
  \begin{equation}\label{eq:eqgraphadjnum}
    \eqgraphsnum{\graph} = \sum_{\substack{u, w \in \rows \\ u < w}}
    \indic{\rowsum{u} = \rowsum{w} \wedge \colsneqnum{u}{w} = 2}
  \end{equation}
  is the number of graphs $\graphadj$ that are yielded from $\graph$ with a
  local swap defined by \textnormal{\findadj{}} such that $\graphadj \equiv
  \graph$. $\indic{\cdot}$ is the indicator function taking value 1 if the
  argument is true, and 0 otherwise, and
  \[
    \colsneqnum{u}{w} = |\{z \in \cols : \dataset(u, z) \neq \dataset(w, z)\}|
  \]
  is the number of column values in $\dataset$ that differ across rows $u$ and
  $w$.
\end{theorem}

\begin{proof}
  Equation (\ref{eq:graphadjnum}) follows from the observation that subtracting
  $\eqgraphsnum{\graph}$, the number of adjacent graphs $\graphadj$ such that
  $\graphadj \equiv \graph$, from $\degree{\graph}$, the total number of
  adjacent graphs $\graphadj$, yields $\graphadjnum{\graph}$, the total number
  of swappable pairs of edges for state $\graph$.

  To show that Equation (\ref{eq:eqgraphadjnum}) correctly computes the
  number of graphs $\graphadj$ such that $\graphadj \equiv \graph$, note
  that this number is equivalent to the number of pairs of edges $(u, v), (w,
  x) \in \edges(\graph)$ such that $(u, x), (w, v) \notin \edges(\graph)$,
  where swapping edges $(u, v)$ and $(w, x)$ leads to an equivalent graph.
  This set of pairs of edges can be written as
  \[
    \spairsgieq = \{((u, v), (w, x)) \in \spairsgi : \forall z \in \cols
    \setminus \{v, x\}, \dataset(u, z) = \dataset(w, z)\},
  \]
  where
  \[
    \spairsgi = \{((u, v), (w, x)) : (u, v), (w, x) \in \edges(\graph), u < w
    \wedge (u, x), (w, v) \notin \edges(\graph)\}.
  \]
  $\eqgraphsnum{\graph}$ can also be expressed as $\eqgraphsnum{\graph} =
  |\eqgraphs|$, where
  \[
    \eqgraphs = \{(u, w) \in \rows : u < w \wedge \rowsum{u} = \rowsum{w}
    \wedge \colsneqnum{u}{w} = 2\}.
  \]
  Hence, it suffices to show that $|\spairsgieq| = |\eqgraphs|$. Just like
  before, we find a bijective function from $\spairsgieq$ to $\eqgraphs$.
  Let $\bijfunc : \spairsgieq \rightarrow \eqgraphs$ be a function defined by
  \[
    \bijfunc((u, v), (w, x)) = (u, w),
  \]
  where $((u, v), (w, x)) \in \spairsgieq$ and $(u, w) \in \eqgraphs$.

  We start with showing the $\bijfunc$ is injective. Let $\spairsgieqelem{1},
  \spairsgieqelem{2} \in \spairsgieq$, where $\spairsgieqelem{1} = ((u_1, v_1),
  (w_1, x_1))$ and $\spairsgieqelem{2} = ((u_2, v_2), (w_2, x_2))$. If
  $\bijfunc(\spairsgieqelem{1}) = \bijfunc(\spairsgieqelem{2})$, then $(u_1,
  w_1) = (u_2, w_2)$, and so $u_1 = u_2$ and $w_1 = w_2$. Because there are
  only two columns where the values are different across rows $u_1 = u_2$ and
  $w_1 = w_2$, we have that $v_1 = v_2$ and $x_1 = x_2$. It follows that
  $\spairsgieqelem{1} = \spairsgieqelem{2}$, and therefore $\bijfunc$ is
  injective.

  We proceed with proving that $\bijfunc$ is surjective. Let $\eqgraphselem{}
  \in \eqgraphs$ such that $\eqgraphselem{} = (u, w)$. Since $\colsneqnum{u}{w}
  = 2$, there are exactly two columns where values in rows $u$ and $w$ differ.
  We denote these two columns as $v$ and $x$. Thus, $\dataset(u, v) \neq
  \dataset(w, v)$ and $\dataset(u, x) \neq \dataset(w, x)$. Further, since
  $\rowsum{u} = \rowsum{w}$, it follows that $\dataset(u, v) \neq \dataset(u,
  x)$ and $\dataset(w, u) \neq \dataset(w, x)$. As such, we either have
  \begin{align*}
    \dataset(u, v) &= 1 \wedge \dataset(u, x) = 0 \wedge \dataset(w, x) = 1
    \wedge \dataset(w, v) = 0 \text{ or } \\
    \dataset(u, v) &= 0 \wedge \dataset(u, x) = 1 \wedge \dataset(w, x) = 0
    \wedge \dataset(w, v) = 1
  \end{align*}
  That is, either $((u, v), (w, x)) \in \spairsgi$ or $((u, x), (w, v)) \in
  \spairsgi$. Without loss of generality, suppose $((u, v), (w, x)) \in
  \spairsgi$. Then, if $\spairsgieqelem{} \in \spairsgieq$ such that
  $\spairsgieqelem{} = ((u, v), (w, x))$, it follows that $\bijfunc$ is
  surjective since $\bijfunc(\spairsgieqelem{}) = \eqgraphselem{}$.

  Therefore, $\bijfunc$ is bijective and $|\spairsgieq| = |\eqgraphs|$.
\end{proof}

To calculate $\spairsnum{\graphadj}{\graph}$, suppose we have a hash map that
contains transactions as keys and the number of transactions equal to the key
as the value. Further, suppose we swapped edges $(i, j), (k, l) \in
\edges(\graph)$. To update the hash map, we would subtract 1 from the number of
transactions equal to $\dataset(i)$ and add 1 to the number of transactions
equal to $\datasetadj(i)$; we would do the same for $k$ as well.
$\spairsnum{\graphadj}{\graph}$ would then be the number of rows equal to
$\datasetadj(i)$ times the number of rows equal to $\datasetadj(k)$.
Computing $\graphadjnum{\graphadj}$ follows immediately from
\cref{thm:graphadjnum}. Specifically,
\begin{equation}
  \graphadjnum{\graphadj} = \graphadjnum{\graph} + \degreedel - \eqgraphsnumdel,
\end{equation}
where $\degreedel = \degree{\graphadj} - \degree{\graphadj}$
\cref{eq:graphadjnum} and $\eqgraphsnumdel$ is computed in
\cref{algo:eqgraphsnumdel}.

\begin{algorithm}
  \caption{\geteqgraphsnumdel}\label{algo:eqgraphsnumdel}
	\DontPrintSemicolon{}
  \SetKwData{RowSumIndicesMap}{sumIndicesM}
  \SetKwData{RowPairNumColsNeqMap}{pairColsNeqM}
  \SetKw{Continue}{continue}
  \SetKwFunction{SortIncreasing}{sortIncreasing}
  \KwIn{0--1 matrix $\dataset$, swapped edges $(i, j), (k, l) \in
    \edges(\graph)$, \RowSumIndicesMap: map where each key is a row sum and
    each value is a list of row indices that have a row sum equal to the key,
    \RowPairNumColsNeqMap: map where each key is a pair of row indices $(u, w)$
    and each value is the number of column values that differ across the pair
    of rows $\colsneqnum{u}{w}$}
  \KwOut{$\eqgraphsnumdel$}

  $\eqgraphsnumdel \leftarrow 0$\;
  \ForEach{$u \in \{i, k\}$}{
    \ForEach{$w \in \RowSumIndicesMap[\rowsum{u}]$}{
      \If{$(u = w) \vee (u = i \wedge w = k) \vee (u = k \wedge w = i)$}{
        \Continue\;
      }
      $u, w \leftarrow \SortIncreasing{u, w}$\;
      $\colsneqnum{u}{w} \leftarrow \RowPairNumColsNeqMap[(u, w)]$\;
      \ForEach{$z \in \{j, l\}$}{
        \If{$\dataset(u, z) = \dataset(w, z)$}{
          $\RowPairNumColsNeqMap[(u, w)]--$\;
        }
        \Else{
          $\RowPairNumColsNeqMap[(u, w)]++$\;
        }
      }
      \If{$\RowPairNumColsNeqMap[(u, w)] = 2 \wedge \colsneqnum{u}{w} \neq
        2$}{
        $\eqgraphsnumdel++$\;
      }
      \ElseIf{$\RowPairNumColsNeqMap[(u, w)] \neq 2 \wedge \colsneqnum{u}{w} =
        2$}{
        $\eqgraphsnumdel--$\;
      }
    }
  }
  \Return{$\eqgraphsnumdel$}
\end{algorithm}
